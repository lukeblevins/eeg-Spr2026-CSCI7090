\documentclass[conference]{IEEEtran}
% \IEEEoverridecommandlockouts
% The preceding line is only needed to identify funding in the first footnote. If that is unneeded, please comment it out.
\usepackage{amsmath,amssymb,amsfonts}
\usepackage{algorithmic}
\usepackage{graphicx}
\usepackage{textcomp}
\usepackage{xcolor}
\usepackage{hyperref}
\usepackage{csquotes}
\usepackage[backend=bibtex,style=ieee]{biblatex}
\addbibresource{references.bib}
\def\BibTeX{{\rm B\kern-.05em{\sc i\kern-.025em b}\kern-.08em
    T\kern-.1667em\lower.7ex\hbox{E}\kern-.125emX}}
\begin{document}

\title{
Multimodal Seizure Prediction Using EEG Time-Series Features and Clinical Metadata
% TODO: Edit GitHub repo and documentation to use a more descriptive name (TBD)
\thanks{Code available at: \url{https://github.com/lukeblevins/mri-imgseg-Spr2026-CSCI7090}}
}

\author{
\IEEEauthorblockN{Luke Blevins}
\IEEEauthorblockA{Department of Computer Science\\
Georgia Southern University\\
Statesboro, United States \\
lb22835@georgiasouthern.edu}
\and
\IEEEauthorblockN{Jacob Rawlins}
\IEEEauthorblockA{Department of Computer Science\\
Georgia Southern University\\
Statesboro, United States \\
jr29003@georgiasouthern.edu}
\and
\IEEEauthorblockN{Arnoldo Vilches-Arteaga}
\IEEEauthorblockA{Department of Computer Science\\
Georgia Southern University\\
Statesboro, United States \\
av05604@georgiasouthern.edu}
}

\maketitle


\begin{abstract}
Millions of people worldwide suffer from a neurological condition known as epilepsy, which causes recurring seizures in those afflicted. These seizures and their symptoms vary widely from person to person, making diagnosis and treatment difficult for medical professionals. Electroencephalography (EEG) recordings are used not only to diagnose but to predict epileptic seizures. Machine Learning algorithms using these recordings have emerged over the years as a useful tool for epileptic seizure prediction. This systematic literature review aims to analyze the accuracy and performance of these models. Code available at: \url{https://github.com/lukeblevins/mri-imgseg-Spr2026-CSCI7090}
\end{abstract}

\begin{IEEEkeywords}
Electroencephalography, epilepsy, deep learning, machine learning
\end{IEEEkeywords}

\section{Introduction}
Epilepsy is considered one of the most common neurological disorders, affecting an estimated 50 million people worldwide \cite{shafiezadeh2024crosspatient}. It causes recurring seizures in those afflicted, which involve moments of involuntary movement and, in some cases, the loss of consciousness \cite{shafiezadeh2024crosspatient}. Although people with epilepsy can live normal lives, they still face serious risks such as an increased chance of disability or even death. \cite{shafiezadeh2024crosspatient} Therefore, the use of machine learning models for seizure prediction would significantly improve patients' quality of life by giving them time to prepare and seek medical attention. 

There are many methods for properly diagnosing epileptic seizures, such as the electroencephalogram (EEG), which measures electrical activity in the brain and identifies abnormalities \cite{shafiezadeh2024crosspatient}. EEG data is also used by machine learning models to predict epileptic seizures. However, the accuracy and performance of these models vary; this paper aims to analyze and review the models used in epileptic seizure prediction.

\section{Methodology}
For this systematic literature review, the scientific databases searched were PubMed, IEEE Xplore, and ScienceDirect. The keywords used in the search included “EEG”, “seizure prediction”, “epilepsy”, “deep learning”, “neural networks”, and “machine learning”. The resulting articles were reviewed by title and abstract to ensure relevance to the topic, and then screened against the inclusion and exclusion criteria.  

\subsection{Inclusion Criteria}
\begin{itemize}
  \item \textbf{1:} Articles published between 2021 and the present date.
  \item \textbf{2:} Articles written in English.
  \item \textbf{3:} Peer-reviewed articles published in a highly recognized journal.
  \item \textbf{4:} Articles that have been cited in numerous other articles.
\end{itemize}

\subsection{Exclusion Criteria}
\begin{itemize}
  \item \textbf{1:} Articles on animal studies rather than human.
  \item \textbf{2:} Articles that did not use EEG data.
  \item \textbf{3:} Articles foucssed primarily on seizure dection opposed to prediction.
  \item \textbf{4:} Articles that are not peer-reviewed.
\end{itemize}

\section{Literature Review}
Accurate epileptic seizure diagnosis remains clinically challenging because seizure events are often unwitnessed or incompletely described. In addition, specialized expertise for analyzing EEG data is limited, and patients may therefore present to non-experts for initial evaluation.

Recent systematic reviews and methodological surveys indicate a growing body of work applying machine learning and deep learning to EEG-based seizure prediction and related epilepsy assessment tasks \cite{han2024epilepsycare,li2025signalsanalysis,vallabhaneni2021eegdecoding}. Han \textit{et al.} \cite{han2024epilepsycare} review AI/ML applications for epilepsy and seizure diagnosis across modalities (including EEG, neuroimaging, wearables, and seizure video) and emphasize that clinical integration depends on generalizability, interpretability, and consistent reporting practices. Complementing this perspective, Li \textit{et al.} \cite{li2025signalsanalysis} provide a methodological synthesis of deep learning for EEG and intracranial EEG (iEEG) analysis and formalize evaluation strategies (subject-specific, mixed-subject, and cross-subject), arguing that cross-subject testing best reflects deployment on previously unseen patients. Vallabhaneni \textit{et al.} \cite{vallabhaneni2021eegdecoding} survey deep learning approaches for EEG decoding and note persistent practical constraints, including hyperparameter sensitivity, computational cost, and limited robustness across subjects and acquisition settings. These themes motivate careful validation design and transparent performance reporting in seizure prediction research.

Multiple studies have shifted towards the use of deep learning architectures designed to identify discriminative features during the preictal phase, the period preceding seizure onset. This literature review analyzes recent systematic reviews as well as experimental studies to identify current methodological practices in the field, highlight gaps and weaknesses, and discuss emerging trends \cite{carmo2024seizureforecast,shafiezadeh2024crosspatient,mourad2025seizurerecognition}.

A strong contribution to this field is the study titled ``Automated algorithms for seizure forecast: a systematic review and meta-analysis'' \cite{carmo2024seizureforecast}. This article provides a systematic review of patient-specific seizure forecasting algorithms to assess reported performance and establish reference points for future developments. The authors discuss the lack of standardization in seizure forecasting research and unclear performance benchmarks, and note the need to distinguish between forecasting and prediction while improving evaluation practices. To investigate these issues, the study conducted a filtered search of scientific databases, removed duplicates, screened results, and performed a meta-analysis. The findings highlight variability in study design and evaluation. In response, the authors propose guidelines for seizure forecast solution design, provide performance benchmarks for seizure forecast algorithms, and emphasize a clearer distinction between seizure forecasting and seizure prediction. The review also reports inconsistent use of probabilistic performance metrics and inconsistent definitions of forecast horizons.

While the previous review highlights the need for standardization of metrics, another key issue is the heavy reliance on patient-specific data. To address this issue, the study titled ``A systematic review of cross-patient approaches for EEG epileptic seizure prediction'' \cite{shafiezadeh2024crosspatient} examines patient-independent testing in order to evaluate more rigorous model assessment practices. The authors searched scientific databases for scalp EEG seizure prediction studies, applied keyword-based filtering, used bias assessment tools, and analyzed selected studies for cross-patient approaches. Through this analysis, the authors concluded that relatively few studies use cross-patient testing and that many studies lack large-scale annotated datasets. Additionally, there is no standardized workflow to validate seizure prediction systems.

In addition to these concerns, questions regarding the effectiveness of models remain. The article ``Machine and deep learning methods for epileptic seizure recognition using EEG data: A systematic review'' \cite{mourad2025seizurerecognition} evaluates the use of EEG data analysis using machine learning and deep learning models. The authors follow a PRISMA-guided process consisting of identification, screening, and inclusion stages. The review identifies several gaps, including the gap between AI-driven technology and clinical application, the need for more interpretable and transparent AI models, the need for larger and more balanced datasets, and the lack of standardized boundaries for the preictal state. The authors also highlight scalability issues, including computational demands of deep learning models and class imbalance that can affect model reliability.

Building upon the previous article, the study titled ``Neural networks for epilepsy detection and prediction with EEG signals: a systematic review'' provides a large scoping analysis of neural-network-based approaches in EEG seizure research. This study provides a broader overview, including seizure prediction as well as seizure detection. Using the PRISMA methodology, 341 studies were selected through inclusion and exclusion criteria. The review analyzes datasets, preprocessing, architectures, and results to identify common pipeline components as well as recurring limitations. Similar to prior reviews, this study highlights reliance on limited datasets, frequent use of patient-specific modeling, and inconsistent evaluation metrics.

Extending the findings of previous systematic reviews, the study ``A Review of Machine Learning and Deep Learning Trends in EEG-Based Epileptic Seizure Prediction'' provides a focused review of EEG-based seizure prediction with an analysis of 149 recent studies. While the previous study emphasizes neural-network approaches broadly, this study analyzes trends across both machine learning and deep learning studies. It reviews dataset usage, preprocessing strategies, validation schemes, and evaluation metrics within seizure prediction pipelines. Consistent with earlier reviews, the authors identify overreliance on limited datasets, inconsistencies in evaluation metrics, and the popularity of convolutional neural network models. The study also notes that repeatedly reported high accuracy may not reflect clinically relevant performance when models are trained and tested under patient-specific or static-dataset conditions.

While systematic reviews provide insight into recurring limitations in seizure prediction research, experimental studies continue to develop new approaches to predict seizures. One example is the study ``Predicting epileptic seizures based on EEG signals using spatial depth features of a 3D-2D hybrid CNN.'' This work proposes a hybrid 3D--2D convolutional neural network to capture spatial and inter-channel relationships in multi-channel EEG data. The model combines 3D and 2D convolutional layers to capture spatial structure and inter-channel correlations without substantially increasing computational complexity. However, limitations described in the systematic reviews remain relevant, as the study is evaluated on static datasets rather than continuous real-time settings and uses patient-specific modeling. As the field advances, stronger evidence for clinical translation will require evaluation on more generalizable data and validation protocols aligned with deployment conditions.

\section{Results}
The proposed integrated framework demonstrates improved predictive performance compared to models using imaging or clinical data alone.

\section{Discussion}
These findings highlight the value of tumor-aware feature extraction and integrated learning for interpretable prognosis in neuro-oncology.


\nocite{*}
\printbibliography


\end{document}
